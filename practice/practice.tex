\documentclass{ximera}
      
\title{Central Limit Theorem}
      
\begin{document}
    
 \begin{abstract}
 Some practice problems regarding the Central Limit Theorem.
 \end{abstract}   
    
      
      
\maketitle
      
 
\begin{enumerate}

\item Consider the following graph:

\begin{onlineOnly}
$$\graph[xmin=-2,xmax=2,ymin=-0.5,ymax=1, panel]{N=e^{-x^2/2}/(2\pi)^{(0.5)}, P=\int_{z_1}^{z_2} N dx, a=-1, b=1, s=1, m=0, z_1=(a-m)/s, z_2=(b-m)/s, 0<y<N\left\{z_1<x<z_2\right\}}$$
\end{onlineOnly}
(This graph allows one to visualize and compute normally distributed probabilities.  Sliders $a$ and $b$ control the left and right endpoints, whereas $m, s$ denote the mean and standard deviation respectively.  Then $P$ is proportion of the normal curve between $a$ and $b$ with $z$-scores $z_1, z_2$ respectively.  A larger version of this graph may be found here: \url{https://www.desmos.com/calculator/qt5u3rb3yo}.)


\begin{sageCell}
from scipy.stats import norm

normdist = norm()

mean=120 # Put the mean of the distribution here
std=30  # Put the standard deviation of the distribution here.
P=0.4  # P is the value such that everything to the left of X is P

X=normdist.ppf(P)*std+mean

print('The area to the left of '+`X`); print(' on the normal curve with mean '+`mean`); print(' and standard deviation '+`std`); print(' is  '+`P`)
\end{sageCell}
(The above SageCell allows one to input a probability/area $P$ and it will find a value $X$ so that the probability/are to the \textbf{LEFT} of $X$ is $P$.)



\begin{problem}
Suppose that Kyle Sumner takes his new boat out into the water to catch yellowfin tuna, whose weights are normally distributed with mean 120 pounds and standard deviation 30 pounds.  

\begin{enumerate}
\item If Kyle catches a yellowfin, what is the probability that he catches a tuna between 120 and 140 pounds? (round to 4 decimal places is necessary) $\begin{prompt}
    P(120\leq X\leq 140)= \answer{0.2475}
  \end{prompt}$ 
  \item  Find a value $X$ so that the probability that Kyle catches a fish at least $X$ pounds is 60\% (round to 4 decimal places is necessary) $X=\answer{112.3996}$
  \item If Kyle catches 36 tuna, what is the probability that the average weight of these tuna will be less than 110 pounds? (round to 4 decimal places is necessary) $\answer{0.0228}.$

\end{enumerate}
\end{problem}


\begin{problem}
The amount of annual snowfall in a certain mountain range is normally distributed with a mean of 109 inches, and a standard deviation of 10 inches
\begin{enumerate}
\item What is the probability that in a 25 year period, the average annual snowfall will be more than 110 inches? (round to 4 decimal places is necessary) $\answer{0.3085}.$
\end{enumerate}

\item Find a value $X$ so that we can be sure that the average annual snowfall for the next 4 years will be at least $X$. (round to 4 decimal places is necessary) $X=\answer{97.3683}.$

\end{problem}

\begin{problem}
Suppose that a class of 25 students takes a multiple choice exam which has 100 problems, and each student has an 80\% chance of answering each question correctly.  Use the Central Limit Theorem to estimate the probability that on average, the class answers at least 77 problems correctly. (round to 4 decimal places is necessary) $X=\answer{0.7734}.$



\end{problem}




\end{enumerate}





 
 
 
 
      






\end{document}
